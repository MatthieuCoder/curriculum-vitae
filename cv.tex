\PassOptionsToPackage{dvipsnames}{xcolor}

\documentclass[10pt,a4paper]{altacv}
\geometry{left=1cm,right=9cm,marginparwidth=6.8cm,marginparsep=1.2cm,top=1.25cm,bottom=1.25cm,footskip=2\baselineskip}

\usepackage[T1]{fontenc}
\usepackage[utf8]{inputenc}
\usepackage[default]{lato}
\usepackage{setspace}
\usepackage{progressbar}
\usepackage{grid-system}

\definecolor{Navy}{HTML}{000090}
\definecolor{SlateGrey}{HTML}{2E2E2E}
\definecolor{LightGrey}{HTML}{666666}
\colorlet{heading}{Navy}
\colorlet{accent}{Navy}
\colorlet{emphasis}{SlateGrey}
\colorlet{body}{LightGrey}

% Change the bullets for itemize and rating marker
% for \cvskill if you want to
\renewcommand{\itemmarker}{{\small\textbullet}}
\renewcommand{\ratingmarker}{\faCircle}


\usepackage[colorlinks=false,pdftitle={Pignolet Matthieu},hidelinks]{hyperref}

\begin{document}

\name{Matthieu Pignolet}
\tagline{}
% \photo{2.8cm}{Globe_High}
\personalinfo{%
  % Not all of these are required!
  % You can add your own with \printinfo{symbol}{detail}
  \email{\href{mailto:matthieu@matthieu-dev.xyz}{matthieu@matthieu-dev.xyz}}
  \email{\href{mailto:m.pignolet@rt-iut.re}{m.pignolet@rt-iut.re}}
  \phone{(+262) 693 47 03 81}
  \mailaddress{6, Impasse Bois de Tan Rouge, Le Tampon}
  \location{La Réunion, France}
  \homepage{\href{http://www.matthieu-dev.xyz}{matthieu-dev.xyz}}
  \linkedin{\href{http://www.linkedin.com/in/matthieu-pignolet-8a8971259/}{Matthieu Pignolet}}
   \github{\href{http://github.com/MatthieuCoder}{@MatthieuCoder}} % I'm just making this up though.

}

\begin{fullwidth}
\makecvheader
\end{fullwidth}

Passionné, Technique, Dévelopeur

\AtBeginEnvironment{itemize}{\small}

\cvsection[sidebar]{Expérience}

\cvevent{Stage à WebCup Campus}{Observation en milieu professionnel}{2018}{Saint-Pierre}{
  \begin{itemize}
    \item Mise en place d'un serveur web pour les utilisateurs 
    \item Assistance aux personnes n'ayant pas d'expérience avec le numérique.
  \end{itemize}
}

\divider{}

\cvevent{Developer's House}{Projet associatif entre amis}{Aout 2019 -- Décembre 2022}{}{
  \begin{itemize}
    \item Mise en place de cluster Kubernetes sur bare-metal
    \item Introduction aux technologies cloud-native
    \item Déploiement de stockage et de loadbalancing grâce à Cloudflare
    \item Développement d'une plateforme d'authentification SSO
  \end{itemize}
}

\cvsection{Projets}

\cvevent{Nova}{Projet personnel}{Octobre 2022 -- Présent}{}{
  \smallskip
  \small{
    Framework pour la création d'applicatifs ("Bots") au sein de la plateforme Discord qui 
    assure la fiabilité de l'application grâce à des brokers et du loadbalancing au sein de 
    Kubernetes pour permettre la mise à jour de l'applicatif sans aucune période de non-disponibilité.
  }
  \smallskip
  \begin{itemize}
    \item Conteneurisation Docker multi-architecture (amd64 / arm).
    \item Développement en Rust et Go
    \item Utilisation du broker NATS
    \item Déploiement sur Kubernetes
    
  \end{itemize}
}

\divider{}

\cvevent{Timothé-rs}{Projet personnel}{Octobre 2021 -- Présent}{}{
  \smallskip
  \small{
    Système de suivi d'emploi du temps pour avertir les elèves d'éventuels changements
  }
  \smallskip
  \begin{itemize}
      \item Utilisation de Rust pour le parsing de fichiers .ics
      \item Déploiement dans Kubernetes au sein d'un cloud
  \end{itemize}
}

\divider{}

\cvevent{libmatpgn}{IUT de la Réunion}{Avril 2023 - Mai 2023}{Saint-Pierre}{
  \smallskip
  \small{
    Librairie utilisée pour parser les fichiers .pgn au sein de la ressource "Analyse et traitement de données structurées" de mon BUT R\&T.
  }
}

\divider{}

\cvevent{Kuizz}{Projet personnel}{Aout 2020 -- Présent}{}{
  \smallskip
  \small{
    Application mobile présentant un flux infini de questions de culture générale.
  }
  \smallskip
  \begin{itemize}
      \item Utilisation de Flutter de gRPC et Rust pour un serveur de jeu
  \end{itemize}
}

\clearpage
\end{document}