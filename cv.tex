\PassOptionsToPackage{dvipsnames}{xcolor}

\documentclass[10pt,a4paper]{altacv}
\geometry{left=1cm,right=9cm,marginparwidth=6.8cm,marginparsep=1.2cm,top=1.25cm,bottom=1.25cm,footskip=2\baselineskip}

\usepackage[T1]{fontenc}
\usepackage[utf8]{inputenc}
\usepackage[default]{lato}
\usepackage{setspace}
\usepackage{progressbar}
\usepackage{grid-system}

\definecolor{Navy}{HTML}{000090}
\definecolor{SlateGrey}{HTML}{2E2E2E}
\definecolor{LightGrey}{HTML}{666666}
\colorlet{heading}{Navy}
\colorlet{accent}{Navy}
\colorlet{emphasis}{SlateGrey}
\colorlet{body}{LightGrey}

% Change the bullets for itemize and rating marker
% for \cvskill if you want to
\renewcommand{\itemmarker}{{\small\textbullet}}
\renewcommand{\ratingmarker}{\faCircle}


\usepackage[colorlinks=false,pdftitle={Pignolet Matthieu},hidelinks]{hyperref}

\begin{document}

\name{Matthieu Pignolet}
\tagline{}
\personalinfo{
  \email{\href{mailto:m@mpgn.dev}{m@mpgn.dev}}
  \email{\href{mailto:m.pignolet@rt-iut.re}{m.pignolet@rt-iut.re}}
  \phone{(+262) 693 47 03 81}
  \homepage{\href{http://puffer.fish/~matthieu}{puffer.fish/\~{}matthieu}}
  \linkedin{\href{https://www.linkedin.com/in/mpgn/}{Matthieu Pignolet}}
   \github{\href{http://github.com/MatthieuCoder}{@MatthieuCoder}}
}

\begin{fullwidth}
\makecvheader
\end{fullwidth}

\cvsection[sidebar]{Expérience}

\cvevent{Exodata}{Alternance durant un BUT R\&T}{Octobre 2023 - Octobre 2024}{}{
  \begin{itemize}
    \item{Développement d'un framework python dans le cadre de systèmes de vérifications et de supervision}
    \item{Développement de systèmes d'intégration continue et de tests continus}
    \item{
     Développement de plusieurs systèmes de vérifications contenant au moins: \\
     \t  * Vérifications de conformité de la réplication dans le cadre de plans de reprise d'activité \\
     \t  * Indexation de systèmes de pare-feu et leurs règles de filtre pour vérifications centralisées \\
     \t  * Vérifications de conformité au sein de plate-formes d'antivirus vendues aux clients
     }
    \item{Développement de système d'indéxation et de stockage de licenses au sein de l'ecosystème Microsoft pour assurer la visibilité}
    \item{Mise en place d'infrastructure basées sur Ansible AWX}
  \end{itemize}
}

\cvevent{Developer's House}{Projet associatif entre amis}{Aout 2019 -- Décembre 2022}{}{
  \begin{itemize}
    \item Mise en place de cluster Kubernetes sur bare-metal
    \item Introduction aux technologies Cloud-native
    \item Déploiement de stockage et de loadbalancing grâce à Cloudflare
    \item Développement d'une plateforme d'authentification SSO
  \end{itemize}
}

\cvsection{Projets}

\cvevent{Orion}{Solution de sd-wan open-source basée sur Wireguard et FRR (BGP)}{2020 - Actuel}{}{
  \begin{itemize}
    \item {Système de ``SD-wan'' inter-étudiant utilisant BGP et Wireguard pour assurer une redondance}
    \item{Mise en place de nombreux services disponibles au sein du réseau (Relais SIP, zones DNS publiques, serveurs SMTP avec relais)}
    \item {Utilisation de la technologie de VPN Wireguard}
    \item {Implémentation de techniques de hole-punching pour les liaisons derrière des NATs}
    \item {Mise en place d'authentification par certificat (mTLS) au serveur de signalisation}
    \item {Système de génération de règles de NAT par nftables développé}
    \item {Génération de configuration BGP pour \href{https://github.com/FRRouting/frr.git}{FRR} et systèmes de rpc via gRPC et Websocket}
  \end{itemize}
}

\cvevent{Lab EVPN proxmox}{Déploiement d'une infrastructure dual-stack basée sur l'evpn proxmox}{2023 - Actuel}{}{
  Ces tests ont abouti par une contribution a Proxmox au niveau d'entregistement DNS dynamiques, ainsi qu'un report de bug au niveau de FRR.
}

\clearpage
\end{document}