\cvsection{Projets personnels}

\begin{cventries}
  \cventry
    {Solution de SD-WAN open-source basée sur Wireguard et FRR}
    {Projet ``Orion'' \href{https://github.com/Orion-network-dev}{\faGithubSquare}}
    {Projet personnel}
    {2020-Actuel}
    {
      \begin{cvitems}
        \item {Système de ``SD-wan'' inter-étudiant utilisant BGP et Wireguard pour assurer une redoncance,
        de nombreux services sont disponibles au sein du réseau (Relais SIP, zones DNS publiques, serveurs SMTP avec relais), 
        }
        \item {Utilisation de la technologie de VPN Wireguard}
        \item {Implémentation de techniques de hole-punching pour les liaisons derrière des NATs}
        \item {Mise en place d'authentification par certificat (mTLS) au serveur de signalisation}
        \item {Système de génération de règles de nat par nftables dévelopé}
        \item {Génération de configuration BGP pour \href{https://github.com/FRRouting/frr.git}{FRR} et systèmes de rpc via gRPC et Websocket}
      \end{cvitems}
    }

  \cventry
    {Système d'édition d'images par webgl}
    {Projet ``nsi-image-editor'' \href{https://github.com/MatthieuCoder/nsi-image-editor}{\faGithubSquare}}
    {Projet de fin d'année}
    {2020-Actuel}
    {
      \begin{cvitems}
        \item {Introduction aux systèmes de shader de fragments et de vertex} 
        \item {Introduction a des APIs graphiques utilisées au sein du web}
        \item {Déploiement sur un bucket AWS \href{https://cdn.mpgn.dev/image-editor/index.html}{cdn.mpgn.dev/image-editor/index.html}}
      \end{cvitems}
    }

  \cventry
    {Système de suivi d'emploi du temps}
    {Projet ``Timothé-rs'' \href{https://github.com/MatthieuCoder/Timothe-rs.git}{\faGithubSquare}}
    {Utilitaire pour la promotion R\&T}
    {2020-2021}
    {
      \begin{cvitems}
        \item {Implémentation d'un arbre de recherche (BTreeMap) selon la date pour une recherche efficace} 
        \item {Utilisation de l'api discord (websocket) pour la mise en place d'un robot discord}
        \item {Déploiement en contenur au sein d'une machine digitalocean}
      \end{cvitems}
    }
  \cventry
    {Système de complétion de mots automatique}
    {Projet ``Gru'' [\href{https://git.puffer.fish/?p=matthieu/gru.git;a=summary}{Git-web}]}
    {Bot discord}
    {2019-Présent}
    {
      \begin{cvitems}
        \item {Implémentation d'un arbre de recherche (Trie) en Rust pour la recherche de mods selon leur début Phonème (Algorithme Lossy)} 
        \item {Utilisation d'une librarie utilisant du DeepLearning pour une traduction Graphème vers Phonème}
        \item {Système de découpe par syllabes (Algorithme Lossy)}
        \item {Déploiement sous forme de stack compose}
      \end{cvitems}
    }
\end{cventries}
